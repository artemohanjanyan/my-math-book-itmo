\documentclass[12pt, a4paper]{article}
\usepackage[utf8]{inputenc} % если ваш файл содержит русский текст, нужно указать кодировку
\usepackage[russian]{babel} % для того, чтобы писать русский текст
\usepackage{amsmath} % для команды equation*
\usepackage{hyperref} % для вставки ссылок
\usepackage{graphicx}
\usepackage[russian]{babel}
\parindent 0pt
\parskip 0pt
\usepackage{amsmath}
\usepackage{amssymb}
\usepackage{array}
\usepackage[left=2.3cm, right=7.3cm, top=1.7cm, bottom=1.7cm, bindingoffset=0cm]{geometry}
\usepackage{hyperref}
\usepackage{graphicx}
\usepackage{float}
\usepackage{enumitem}
\usepackage{array}
\usepackage{subcaption}
\usepackage{multicol}
\usepackage{fancyhdr} 
\usepackage{extramarks}
\usepackage{todonotes}
\usepackage[usenames,dvipsnames]{color}
\usepackage{titlesec}
\graphicspath{{pictures/}}
\usepackage{lipsum}                     % Dummytext
\usepackage{xargs}                      % Use more than one optional parameter in a new commands
\usepackage[pdftex,dvipsnames]{xcolor} 
\title{Математический анализ.}
\author{Автор конспекта: Терентьев Михаил, Константин Ништ}
\date{\today}

\pagestyle{fancy}
\fancyhf{}
\lhead{Билет № 100500}
\chead{Математический анализ}
\rhead{\thepage}
\lfoot{Автор конспекта: Терентьев Михаил }
\cfoot{}
\rfoot{\today}
\renewcommand\headrulewidth{0.4pt}
\renewcommand\footrulewidth{0.4pt}
\newcommand{\nl}{\newline}
\newcommand{\intba}{\int^b_a}
\begin{document} 
	\section{Теорема Кантора о равномерной непрерывности} 
	$f : X \rightarrow Y$ непрерывно на Х, Х компактно. \nl
	Тогда f равномерно непрерывно 
	 \section{Теорема Брауэра}$f : [0; 1]^n \in \mathbb{R}^n \rightarrow [0; 1]^n$ \nl
	f непрерывна. \nl
	Тогда : $\exists x \in [0; 1]^n : f(x) = x$ 
	\section{Теорема о свойствах неопределённого интеграла}
	Пусть f и g имеют первообразную на <a; b>, $\alpha , \beta \in \mathbb{R} \newline$
	Тогда : $\newline$
	1. $\int f + g = \int f + \int g \newline$
	2. $\int \alpha f = \alpha  \int f \newline$
	3. $\varphi : <c; d> \rightarrow <a; b>$ дифф $\newline$
	$\int f(\varphi(t)) \varphi ' (t)dt = \int f(x)dx | x = \varphi(t) \newline$
	4. $\int f(\alpha x + \beta)dx = \frac{1}{\alpha} F (\alpha x + \beta) + C$\\
	5. f и g дифференцируемы на <a; b> f'g имеет первообразную. Тогда : $\newline$
	fg' имеет первообразную и $\int fg' = fg - \int f'g $
	 \section{Теорема о среднем}l
	$f \in C[a;b]; \exists c \in [a;b] : \int^b_a f(x)dx = f(c) \cdot (b-a) l$
	\subsubsection{Монотонность} 
	$f, g \in C[a; b]; f<=g \nl$
	Тогда $\int^b_a f <= \int^b_a g$ 
	 \subsubsection{Теорема Барроу}
	$\Phi : [a;b] \rightarrow \mathbb{R}, f \in [a;b] \nl$
	$\Phi(x) = \int_a^x f$ интеграл с переменным верхним пределом. \nl
	$\Phi(a) = 0$
\end{document}